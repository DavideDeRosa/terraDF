\documentclass[a4paper,12pt]{report}
\usepackage[utf8]{inputenc}
\usepackage[T1]{fontenc}
\usepackage{geometry}
\usepackage[hidelinks]{hyperref}
\usepackage{graphicx}
\usepackage{titlesec}
\usepackage{fancyhdr}
\usepackage{lipsum}
\usepackage{xcolor}
\usepackage{tcolorbox}
\usepackage{pdfpages}

\pagestyle{fancy}
\fancyhf{}
\fancyhead[L]{\textit{\chaptertitle}}
\fancyhead[R]{\thepage}
\renewcommand{\headrulewidth}{0.000009pt}

\fancypagestyle{plain}{
  \fancyhf{}
  \fancyhead[L]{\textit{\chaptertitle}}
  \fancyhead[R]{\thepage}
  \renewcommand{\headrulewidth}{0.4pt}
}

\newcommand{\chaptertitle}{}
\renewcommand{\chaptermark}[1]{
  \markboth{#1}{}
  \renewcommand{\chaptertitle}{#1}
}

\geometry{a4paper, margin=1in}

\titleformat{\chapter}[block]
  {\normalfont\LARGE\bfseries}{\thechapter.}{1em}{}
\titleformat{\section}[block]
  {\normalfont\Large\bfseries}{\thesection.}{1em}{}

\hypersetup{
    colorlinks=false,
    linkcolor=blue,
    filecolor=magenta,      
    urlcolor=cyan,
    pdftitle={Document},
    bookmarks=true,
    pdfpagemode=FullScreen,
}

\renewcommand{\contentsname}{Indice}

\begin{document}

\begin{titlepage}
    \centering
    \vspace*{0.1cm}

    \Huge
    \textbf{UNIVERSITÀ DI BOLOGNA}

    \vspace{1cm}
    \Large
    Dipartimento di Informatica - Scienza e Ingegneria \\
    Corso di Laurea Magistrale in Informatica \\\vspace{1cm}
    Progetto corso di \href{https://www.unibo.it/it/studiare/dottorati-master-specializzazioni-e-altra-formazione/insegnamenti/insegnamento/2024/479039}{Digital Forensics} \\
    A.A. 2024/25

    \vspace{5.5cm}
    \textbf{\LARGE 23Bottles S.p.A. vs Partolini S.r.l.}\\\vspace{0.3cm}
    \Large Risposta a parte Attrice

    \vfill

    \vfill

    \large
    Davide De Rosa \hfill Matricola: 0001186536\\
    Marco Coppola \hfill Matricola: 0001170490\\
    Valerio Pio De Nicola \hfill Matricola: 0001170425\\
\end{titlepage}

\tableofcontents
\newpage

\chapter{Risposta a Parte Attrice}
\section{Premessa}
La presente risposta intende evidenziare, alla luce delle analisi forensi condotte, le criticità e le incongruenze presenti nel documento presentato dalla parte attrice, nonché riaffermare la regolarità delle operazioni svolte da Partolini S.r.l. e l'assenza di elementi probatori sufficienti a dimostrare un comportamento scorretto o un inadempimento contrattuale.

\section{Risposte ai Punti Sollevati dalla Parte Attrice}

\begin{itemize}
    \item \textbf{Autenticità e Validità delle Comunicazioni Ufficiali:}  
    la parte attrice sostiene che tutte le email dichiarate da 23Bottles siano state rinvenute e ne testimonia l'autenticità. Tuttavia, va sottolineato che le comunicazioni ufficiali scambiate tra 23Bottles e Partolini, registrate con date e orari confermati dai server di posta elettronica, rappresentano il nucleo centrale della trattativa commerciale. Le ulteriori email, scambiate da indirizzi personali e caratterizzate da firme e linguaggi criptici, si riferiscono a dinamiche interne e non incidono sulla validità delle operazioni logistiche eseguite da Partolini. In questo senso, il documento prodotto dalla parte attrice tende a confondere dati di natura interna con quelli relativi alla transazione ufficiale.

    \item \textbf{Recupero e Decodifica di Email Cancellate e Criptate:}  
    la parte attrice evidenzia il recupero di email cancellate o criptate, suggerendo una presunta attività condotta in modo non trasparente. È importante precisare che l'utilizzo di cifratura -- come il cifrario di Cesare -- in alcuni scambi tra Giulia Zingaro e Giuseppe Pavan è da interpretarsi come un'azione volta probabilmente a proteggere la riservatezza di comunicazioni interne, e non come prova di comportamenti illeciti. Tali comunicazioni non rientrano nell'ambito delle operazioni contrattuali ufficiali e non possono essere considerate elementi determinanti per attribuire responsabilità a Partolini S.r.l.

    \item \textbf{Assenza dei Documenti PDF di Presa in Consegna e del Preventivo:}  
    il documento accusatorio sostiene che il file \textbf{Consegna.pdf} relativo alla presa in consegna della merce e il preventivo inviato da Partolini non siano stati rinvenuti nel computer della società. È nostro parere che l'assenza di tali file sul dispositivo in questione non costituisce in alcun modo una prova dell'inadempimento contrattuale da parte di Partolini. La gestione documentale, infatti, prevede che alcuni file possano essere conservati in sistemi o archivi separati, conformemente alle procedure interne, senza pregiudicare la regolarità delle operazioni.
\end{itemize}

\section{Conclusioni Finali}
Alla luce degli elementi sopra esposti e delle analisi forensi effettuate, si ribadisce che in assenza di ulteriori prove chiare e verificabili, si ritiene che il materiale analizzato non possa in alcun modo attribuire a Partolini S.r.l. responsabilità per eventuali errori nella gestione del trasporto, confermando la regolarità delle operazioni svolte dall’azienda.



\end{document}