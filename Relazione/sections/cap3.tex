\section{PC Portatile Compaq}
In conformità all’autorizzazione del Giudice per il sequestro del materiale informatico appartenente all’azienda Partolini, questa sezione descrive il processo di analisi del materiale confiscato.\\
Tra gli elementi sequestrati vi è il computer portatile utilizzato dalla segretaria, figura chiave nell’intermediazione tra l’azienda Partolini e 23Bottles.\vspace{14pt}\\
Per acquisire l’immagine forense del computer, abbiamo inizialmente smontato il dispositivo ed estratto il disco. Abbiamo successivamente collegato il write blocker -- modello \textit{Tableau TK8U} --  al computer e successivamente il disco al write blocker, seguendo la sequenza prestabilita. Una volta alimentati i dispositivi, abbiamo avviato \textbf{FTK Imager}\footnote{FTK Imager, strumento gratuito fornito da \textit{AccessData}, offre diverse funzionalità, tra cui la possibilità di generare un valore hash per il dispositivo sorgente, creare un’immagine forense e successivamente calcolare un altro valore hash per verificare l’integrità del processo, assicurando l’assenza di modifiche al dispositivo originale.}.\vspace{14pt}\\
Abbiamo eseguito una verifica preliminare dell’integrità del dispositivo tramite FTK Imager. Il valore hash ottenuto ha confermato che non erano state apportate modifiche al dispositivo sorgente, permettendoci così di procedere con la creazione dell’immagine forense.\vspace{14pt}\\
Per il formato dell’immagine, abbiamo scelto il \textbf{formato E01}, considerato il più adatto alle esigenze investigative del caso.

\section{Chiavetta USB}
La chiavetta USB fornita da 23Bottles è stata acquisita utilizzando FTK Imager come volume logico, permettendoci di ottenere anche i diversi Hash dei documenti presenti al suo interno.\vspace{14pt}\\
Viene ignorato il contenuto della cartella \_\_MACOSX, directory di sistema aggiunta automaticamente da MacOS quando si comprime o archivia un insieme di file.