L’azienda \textbf{23Bottles S.p.A.}, d’ora in poi \textit{23Bottles}, decide di presentare istanza di ricorso nei confronti dell’azienda di trasporti \textbf{Partolini S.r.l.}, d’ora in poi \textit{Partolini}, sostenendo che quest’ultima non abbia garantito il servizio di trasporto pattuito contrattualmente. Tale supposizione nasce da un recente scambio di e-mail tra 23Bottles e uno dei clienti, il quale afferma di aver ricevuto da Partolini un numero inferiore di unità dei prodotti di 23Bottles rispetto a quanto richiesto.\vspace{14pt}\\
L’azienda 23Bottles ha inoltre depositato al momento della presentazione della richiesta di ricorso, una \textit{memoria USB} contenente le e-mail scambiate con il cliente e le fatture fornite da Partolini. Questo materiale costituisce \textit{patrimonio aziendale riservato}, ed è l’oggetto del ricorso.\vspace{14pt}\\
Il Giudice ha autorizzato il sequestro del materiale informatico, qualsiasi supporto in grado di immagazzinare dati, in possesso dall’azienda Partolini. Si deve analizzare il materiale sequestrato, ossia il computer portatile detenuto dalla segretaria, che fungeva da tramite di 23Bottles e lo spedizioniere dell’azienda Partolini.\vspace{14pt}\\
Il materiale sequestrato è stato fisicamente consegnato ai consulenti tecnici delle parti. Il contenuto della memoria USB è reperibile presso il sito \textit{Virtuale} del corso.\vspace{14pt}\\
Il Giudice ha posto il seguente quesito, al quale i consulenti delle parti e del Giudice sono chiamati a rispondere:
\begin{quote}
    \textit{"Considerando che oggigiorno le aziende si avvalgono di una serie di
    servizi per la gestione delle proprie attività. Queste aziende possono essere
    vittime di frodi: in queste situazioni, investono risorse inanziarie in servizi
    che promettono beneici ma, al contrario, causano danni all’attività. 
    Si chiede agli specialisti forensi di veriicare la veridicità delle
    aermazioni di parte attrice sulla base del materiale depositato e quello
    sequestrato."}
\end{quote}