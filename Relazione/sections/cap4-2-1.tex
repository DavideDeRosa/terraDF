\begin{center}
    \renewcommand{\arraystretch}{1.5}
    \begin{tabular}{|c|c|}
        \hline
        \textbf{Modello} & 500GB Seagate Laptop Thin \\
        \hline
        \textbf{Codice Seriale} & W625FZAT \\
        \hline
        \textbf{MD5 Checksum} & 808b681e557864a91568a4d6de42f2c9 \\
        \hline
        \textbf{SHA-1 Checksum} & 093eeddb84cf7c73d63ffc83f06244f3d7a2c6b7 \\
        \hline
        \textbf{Size (bytes)} & 500107862016 \\
        \hline
        \textbf{Numero di Partizioni} & 4 \\
        \hline  
    \end{tabular}
\end{center}

\vspace{3pt}
\subsubsection{Partizione 1}
\begin{center}
    \renewcommand{\arraystretch}{1.5}
    \begin{tabular}{|c|c|}
        \hline
        \textbf{Partizione} & Vol1 \\
        \hline
        \textbf{Descrizione} & Non allocata \\
        \hline
        \textbf{Settore iniziale} & 0 \\
        \hline
        \textbf{Lunghezza in settori} & 2048 \\
        \hline
    \end{tabular}
\end{center}

\vspace{3pt}
\subsubsection{Partizione 2}
\begin{center}
    \renewcommand{\arraystretch}{1.5}
    \begin{tabular}{|c|c|}
        \hline
        \textbf{Partizione} & Vol2 \\
        \hline
        \textbf{Descrizione} & NTFS/ExFAT \\
        \hline
        \textbf{Settore iniziale} & 2048 \\
        \hline
        \textbf{Lunghezza in settori} & 204800 \\
        \hline
    \end{tabular}
\end{center}


\vspace{3pt}
\subsubsection{Partizione 3}
\begin{center}
    \renewcommand{\arraystretch}{1.5}
    \begin{tabular}{|c|c|}
        \hline
        \textbf{Partizione} & Vol3 \\
        \hline
        \textbf{Descrizione} & NTFS/ExFAT \\
        \hline
        \textbf{Settore iniziale} & 206848 \\
        \hline
        \textbf{Lunghezza in settori} & 959339537 \\
        \hline
    \end{tabular}
\end{center}

\vspace{3pt}
\subsubsection{Partizione 4}
\begin{center}
    \renewcommand{\arraystretch}{1.5}
    \begin{tabular}{|c|c|}
        \hline
        \textbf{Partizione} & Vol4 \\
        \hline
        \textbf{Descrizione} & Non allocata \\
        \hline
        \textbf{Settore iniziale} & 959546385 \\
        \hline
        \textbf{Lunghezza in settori} & 17226783 \\
        \hline
    \end{tabular}
\end{center}

