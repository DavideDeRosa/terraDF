
L'indagine ha seguito tre fasi distinte:

\begin{enumerate}
    \item \textbf{Acquisizione USB}: inizialmente, è stata condotta un'analisi dei dati contenuti nella memoria USB fornita da 23Bottles, al fine di ottenere una panoramica completa degli eventi e delle comunicazioni. Tale attività ha previsto la sintesi delle email e l'analisi degli eventi associati. Durante questa fase, sono stati identificati gli indirizzi email di rilevanza ai fini della dinamica investigativa. Inoltre, si è resa necessaria l'analisi di email cifrate, che sono state successivamente decifrate per permettere un'indagine approfondita dei contenuti. Infine, è stato possibile ricostruire gli eventi in una linea temporale, garantendo una comprensione dettagliata della sequenza degli avvenimenti e delle interazioni tra le parti coinvolte.
    \item \textbf{Acquisizione HDD}: successivamente, è stata eseguita un'acquisizione forense dell'hard disk del computer utilizzato dalla segretaria dell'azienda Partolini, per garantire l'integrità dei dati raccolti e prevenire la contaminazione delle prove.
    \item \textbf{Analisi comparativa}: infine, è stata condotta un'analisi comparativa dei dati, confrontando le informazioni presenti nella memoria USB con quelle acquisite dal computer di Partolini. L'obiettivo di questa fase è stato quello di identificare corrispondenze o discrepanze, al fine di confermare o confutare le affermazioni avanzate da 23Bottles.
\end{enumerate}
Questo approccio ha permesso di condurre un'analisi forense rigorosa e completa, fornendo una panoramica dettagliata dei fatti e delle circostanze oggetto della controversia legale.
