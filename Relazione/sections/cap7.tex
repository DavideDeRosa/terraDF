Di seguito vengono illustrati i diversi risultati ottenuti durante l'indagine forense.
\section{Analisi USB}
All'interno della memoria USB depositata dall'azienda 23Bottles sono stati rinvenuti diversi file, come annunciato in precedenza:
\begin{itemize}
    \item \textbf{13 email}, tra cui:
        \begin{itemize}
            \item 7 email scambiate tra 23Bottles e Partolini. I due indirizzi con i quali sono avvenute le comunicazioni sono \textit{andrea.zanatta@TTBottles.com} e \\ \textit{giulia.zingaro@partolini.com} [\ref{email1}] [\ref{email2}] [\ref{email4}] [\ref{email6}] [\ref{email9}] [\ref{email11}] [\ref{email13}]
            \item 2 email scambiate tra Giulia Zingaro e Giuseppe Pavan, con email codificate in quasi tutte le comunicazioni. I due indirizzi con i quali sono avvenute le comunicazioni sono \textit{giulia.zingaro@partolini.com} e \textit{giuseppe.pavan74@gmail.com} [\ref{email8}] [\ref{email12}]
            \item 3 email scambiate tra Sara Minto e Giuseppe Pavan. I due indirizzi con i quali sono avvenute le comunicazioni sono \textit{sara.minto80@gmail.com} e \\ \textit{giuseppe.pavan74@gmail.com} [\ref{email3}] [\ref{email5}] [\ref{email10}]
            \item 1 email corrotta, alla quale si può risalire tramite la risposta ricevuta successivamente. I due indirizzi con i quali è avvenuta la comunicazione sono \textit{andrea.zanatta@TTBottles.com} e \textit{giulia.zingaro@partolini.com} [\ref{email7}]
        \end{itemize} 
    \item \textbf{2 file}, tra cui:
        \begin{itemize}
            \item Un documento chiamato \textbf{Preventivo} [\ref{preventivo}] con estenzione \textit{.docx}, nel quale è presente il preventivo per la consegna effettuata da Partolini per 23Bottles
            \item Un documento chiamato \textbf{Consegna} [\ref{consegna}] con estenzione \textit{.pdf}, nel quale è presente la conferma di ricezione dell'ordine
        \end{itemize}
\end{itemize}
\section{Analisi HDD}
Tramite la ricerca per codici hash, email e keyword viene evidenziata la presenza di tutte le email depositate da 23Bottles.\vspace{14pt}\\
Dall’esame delle comunicazioni emerse durante l’indagine, è stato possibile quindi ricostruire una serie di eventi che offrono una visione più chiara delle dinamiche tra 23Bottles e Partolini, oltre che di alcune interazioni interne a quest’ultima.\vspace{14pt}\\
Le prime email tra Andrea Zanatta, rappresentante di 23Bottles, e Giulia Zingaro, referente di Partolini, sembrano rientrare in un normale contesto di trattativa commerciale. In particolare, la richiesta iniziale di informazioni [\ref{email1}] e la successiva conferma di interesse per una collaborazione [\ref{email2}], avvenute il 4 ottobre 2023, indicano una fase di avvio regolare della contrattazione tra le parti.\\
Tuttavia, accanto a queste comunicazioni ufficiali, emergono parallelamente altri scambi di email tra indirizzi personali che suggeriscono una gestione meno trasparente della vicenda.
In particolare, alle ore 19:59:25 GMT dello stesso giorno, viene inviata un’email con oggetto “Solito” dall’indirizzo \textit{sara.minto80@gmail.com} a \textit{giuseppe.pavan74@gmail.com} [\ref{email3}].\\
Il messaggio fa riferimento a un incontro in un luogo abituale per discutere di “novità importanti”. È rilevante notare che la firma riportata è “CZ”, sebbene non corrisponda al nome associato all’indirizzo email. Questo schema si ripete in ulteriori scambi successivi.\vspace{14pt}\\
Il 5 ottobre 2023, alle 10:00:25 GMT, Andrea Zanatta scrive a Giulia Zingaro per esprimere apprezzamento per la collaborazione in corso e per richiedere preventivi dettagliati sui servizi discussi [\ref{email4}]. La risposta di Giulia Zingaro, inviata alle 10:20:12 GMT, contiene il documento richiesto. [\ref{email6}].\\
Poco prima, alle 10:02:12 GMT, ovvero due minuti dopo la ricezione dell’email di Zanatta, Sara Minto invia a Giuseppe Pavan un altro messaggio con oggetto “Solito”, scrivendo solamente "Bingo" firmato "CZ". [\ref{email5}].\vspace{14pt}\\
Il 10 ottobre 2023, alle 15:18 GMT, Andrea Zanatta conferma via email un ordine di 100 pallet, facendo riferimento a un accordo telefonico precedente. Poco dopo, alle 16:54:04 GMT, Giulia Zingaro scrive un’email codificata utilizzando il cifrario di Cesare a Giuseppe Pavan, sottolineando l’importanza della discrezione e richiamando ad una maggiore attenzione. [\ref{email8}].\\
Nello stesso giorno, viene rinvenuto un documento di presa in consegna della merce, firmato da Marco Rizzoli e contenuto tra i materiali informatici depositati da 23Bottles [\ref{consegna}]. La data del documento coincide con quella dell’ultima email con oggetto “Solito” inviata da Giuseppe Pavan a Sara Minto, in cui, firmandosi “MR”, fa riferimento a un brindisi per celebrare il “lavoro svolto” [\ref{email10}].\vspace{14pt}\\
Ulteriori dettagli emergono in seguito al reclamo presentato da Andrea Zanatta il 20 ottobre 2023 [\ref{email11}]. In questa comunicazione, il rappresentante di 23Bottles segnala che un cliente non ha ricevuto l’intera fornitura prevista e sollecita un’indagine per chiarire le cause del problema. Propone inoltre di risolvere la questione con una spedizione urgente della merce mancante per ridurre al minimo l’impatto sulle attività aziendali.\\
Nello stesso giorno, alle 15:25:49 GMT, Giulia Zingaro scrive un’email a Giuseppe Pavan, il cui contenuto decodificato -- anche qui con cifrario di Cesare -- lascia intendere la possibilità di problematiche tra i due dipendenti. Nel messaggio, Zingaro si lamenta infatti di aver consigliato a Pavan di agire con discrezione e si rammarica di dover ora trovare una soluzione alla situazione [\ref{email12}].\vspace{14pt}\\
Infine, il primo dicembre alle 08:31:34 GMT, Andrea Zanatta invia un’ultima email a Giulia Zingaro per sollecitare una risposta alla sua precedente richiesta. Esprime la delusione per la mancata comunicazione e sottolinea che, in assenza di chiarimenti immediati, 23Bottles prenderà provvedimenti legali [\ref{email13}].\vspace{14pt}\\
Inoltre, alla luce dei dati raccolti, si evidenzia come la sezione USB Device Attached di Autopsy abbia mostrato il collegamento, in data 01/05/2023 alle ore 9:52:36 GMT, di un hard disk esterno [\ref{wd}]. Tale collegamento, unitamente agli eventi registrati nei log di sistema, ha consentito di ricostruire una serie di operazioni che, seppur anomale, non risultano direttamente riconducibili ad un comportamento scorretto da parte di Partolini.\vspace{14pt}\\
In particolare, l’analisi dei log di sistema in \texttt{C:/Windows/System32/winevt/Logs/} ha rivelato numerosi cambi di data e ora effettuati dall’utente Laura, con variazioni temporali significative (ad es. un salto di 21 anni e retrocessi di mesi e giorni). Inoltre, l’esame del file \textbf{Security.evtx} ha registrato vari accessi al sistema, con evidenza di:
\begin{itemize}
    \item Accessi fisici (Tipo 2) effettuati tramite la schermata di login.
    \item Accessi remoti (Tipo 3) e login effettuati dai servizi di Windows (Tipo 5).
\end{itemize}
Un ulteriore elemento critico è rappresentato dal file \textbf{Important Document.zip} presente nella directory \texttt{C:/Users/Laura/Desktop}.\\ L’alta entropia del file (misurata a circa 7.999983 bits per byte) indica una cifratura totale, coerente con l’utilizzo di algoritmi di full-text encryption (ad es. VeraCrypt o TrueCrypt) supportati da Windows tramite la cifratura basata sulle informazioni di login. Nonostante i tentativi di decifrazione tramite strumenti come \texttt{hashcat} e vari approcci di boot del disco in ambiente virtuale, il file è rimasto inaccessibile.
Questa inaccessibilità implica che il contenuto di \textbf{Important Document.zip} non può essere analizzato e, di conseguenza, non fornisce ulteriori elementi probatori utili a supportare le tesi della parte attrice. \\In assenza di evidenze chiare e verificabili provenienti da tale file, non è possibile ricostruire con piena affidabilità eventuali discrepanze o errori nella gestione del trasporto da parte di Partolini.\vspace{14pt}\\
Alla luce di questi elementi, si conferma che il materiale analizzato non fornisce basi sufficienti per attribuire a Partolini responsabilità per eventuali errori nella gestione del trasporto, rafforzando così la posizione difensiva dell'azienda.