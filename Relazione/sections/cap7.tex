Di seguito vengono illustrati i diversi risultati ottenuti durante l'indagine forense.
\section{Analisi USB}
All'interno della memoria USB depositata dall'azienda 23Bottles sono stati rinvenuti diversi file, come annunciato in precedenza:
\begin{itemize}
    \item \textbf{13 email}, tra cui:
        \begin{itemize}
            \item 7 email scambiate tra 23Bottles e Partolini. Le due email con la quale sono avvenute le comunicazioni sono \textit{andrea.zanatta@TTBottles.com} e \\ \textit{giulia.zingaro@partolini.com} [\ref{email1}] [\ref{email2}] [\ref{email4}] [\ref{email6}] [\ref{email9}] [\ref{email11}] [\ref{email13}]
            \item 2 email scambiate tra Giulia Zingaro e Giuseppe Pavan, con email codificate in quasi tutte le comunicazioni. Le due email con la quale sono avvenute le comunicazioni sono \textit{giulia.zingaro@partolini.com} e \textit{giuseppe.pavan74@gmail.com} [\ref{email8}] [\ref{email12}]
            \item 3 email scambiate tra Sara Minto e Giuseppe Pavan. Le due email con la quale sono avvenute le comunicazioni sono \textit{sara.minto80@gmail.com} e \\ \textit{giuseppe.pavan74@gmail.com} [\ref{email3}] [\ref{email5}] [\ref{email10}]
            \item 1 email corrotta, alla quale si può risalire tramite la risposta ricevuta successivamente. Le due email con la quale è avvenuta la comunicazione sono \textit{andrea.zanatta@TTBottles.com} e \textit{giulia.zingaro@partolini.com} [\ref{email7}]
        \end{itemize} 
    \item \textbf{2 file}, tra cui:
        \begin{itemize}
            \item Un documento chiamato \textbf{Preventivo} [\ref{preventivo}] con estenzione \textit{.docx}, nel quale è presente il preventivo per la consegna effettuata da Partolini per 23Bottles
            \item Un documento chiamato \textbf{Consegna} [\ref{consegna}] con estenzione \textit{.pdf}, nel quale è presente la conferma di ricezione dell'ordine
        \end{itemize}
\end{itemize}
\section{Analisi HDD}
Tramite la ricerca per codici Hash e la ricerca per email e keyword viene evidenziata la presenza di tutte le email depositate da 23Bottles.\vspace{14pt}\\
Dall’esame delle comunicazioni emerse durante l’indagine, è stato possibile quindi ricostruire una serie di eventi che offrono una visione più chiara delle dinamiche tra 23Bottles e Partolini, oltre che di alcune interazioni interne a quest’ultima.\vspace{14pt}\\
Le prime email tra Andrea Zanatta, rappresentante di 23Bottles, e Giulia Zingaro, referente di Partolini, sembrano rientrare in un normale contesto di trattativa commerciale. In particolare, la richiesta iniziale di informazioni [\ref{email1}] e la successiva conferma di interesse per una collaborazione [\ref{email2}], avvenute il 4 ottobre 2023, indicano una fase di avvio regolare della contrattazione tra le parti.\\
Tuttavia, accanto a queste comunicazioni ufficiali, emergono parallelamente altri scambi di email tra indirizzi personali che suggeriscono una gestione meno trasparente della vicenda.
In particolare, alle ore 19:59:25 GMT dello stesso giorno, viene inviata un’email con oggetto “Solito” dall’indirizzo \textit{sara.minto80@gmail.com} a \textit{giuseppe.pavan74@gmail.com} [\ref{email3}].\\
Il messaggio fa riferimento a un incontro in un luogo abituale per discutere di “novità importanti”. È rilevante notare che la firma riportata è “CZ”, sebbene non corrisponda al nome associato all’indirizzo email. Questo schema si ripete in ulteriori scambi successivi.\vspace{14pt}\\
Il 5 ottobre 2023, alle 10:00:25 GMT, Andrea Zanatta scrive a Giulia Zingaro per esprimere apprezzamento per la collaborazione in corso e per richiedere preventivi dettagliati sui servizi discussi [\ref{email4}]. La risposta di Giulia Zingaro, inviata alle 10:20:12 GMT, contiene il documento richiesto, che però non è stato trovato nei dispositivi informatici di Partolini [\ref{email6}].\\
Poco prima, alle 10:02:12 GMT, ovvero due minuti dopo la ricezione dell’email di Zanatta, Sara Minto invia a Giuseppe Pavan un altro messaggio con oggetto “Solito”, scrivendo solamente un sospetto "Bingo" firmato "CZ". [\ref{email5}].\vspace{14pt}\\
Il 10 ottobre 2023 la situazione si complica ulteriormente. Alle 15:18 GMT, Andrea Zanatta conferma via email un ordine di 100 pallet, facendo riferimento a un accordo telefonico precedente. Poco dopo, alle 16:54:04 GMT, Giulia Zingaro scrive un’email codificata a Giuseppe Pavan, sottolineando l’importanza della discrezione e richiamando un’esperienza fallimentare con un’altra azienda [\ref{email8}]. Il messaggio include la comunicazione ricevuta da Zanatta e suggerisce che la gestione della vicenda si stesse sviluppando su due livelli: uno ufficiale e uno più informale.\\
Nello stesso giorno, viene rinvenuto un documento di presa in consegna della merce, firmato da Marco Rizzoli e contenuto tra i materiali informatici depositati da 23Bottles [\ref{consegna}]. La data del documento coincide con quella dell’ultima email con oggetto “Solito” inviata da Giuseppe Pavan a Sara Minto, in cui, firmandosi “MR”, fa riferimento a un brindisi per celebrare il “lavoro svolto” [\ref{email10}].\vspace{14pt}\\
Ulteriori dettagli emergono in seguito al reclamo presentato da Andrea Zanatta il 20 ottobre 2023 [\ref{email11}]. In questa comunicazione, il rappresentante di 23Bottles segnala che un cliente non ha ricevuto l’intera fornitura prevista e sollecita un’indagine per chiarire le cause del problema. Propone inoltre di risolvere la questione con una spedizione urgente della merce mancante per ridurre al minimo l’impatto sulle attività aziendali.\\
Nello stesso giorno, alle 15:25:49 GMT, Giulia Zingaro scrive un’email a Giuseppe Pavan, il cui contenuto decodificato lascia intendere che dietro ai problemi logistici possano esserci decisioni intenzionali o una gestione volutamente inefficace. Nel messaggio, Zingaro si lamenta infatti di aver consigliato a Pavan di agire con discrezione e si rammarica di dover ora trovare una soluzione alla situazione [\ref{email12}].\vspace{14pt}\\
Infine, il primo dicembre alle 08:31:34 GMT, Andrea Zanatta invia un’ultima email a Giulia Zingaro per sollecitare una risposta alla sua precedente richiesta. Esprime la delusione per la mancata comunicazione e sottolinea che, in assenza di chiarimenti immediati, 23Bottles prenderà provvedimenti legali [\ref{email13}].\vspace{14pt}\\
Nel complesso, questi elementi, sebbene frammentari, delineano un quadro in cui le comunicazioni personali sembrano aver giocato un ruolo chiave nella gestione degli accordi. L’utilizzo di indirizzi email non aziendali, firme non corrispondenti e un linguaggio criptico suggeriscono un tentativo di celare le reali intenzioni e i reali rapporti tra i soggetti coinvolti.


finire di aggiungere le nostre scoperte (live boot + registri + log) + rileggere tutto e parlare nello specifico delle mail criptate