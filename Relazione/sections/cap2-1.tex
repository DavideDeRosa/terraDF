La presente relazione fornisce un’analisi dettagliata delle criticità emerse nella documentazione presentata dalla controparte durante il procedimento investigativo. L’obiettivo è identificare eventuali discrepanze o incongruenze che potrebbero incidere sulla validità delle prove. Il documento è strutturato per esaminare in modo approfondito ogni elemento contestato, offrendo una valutazione basata su prove concrete.
Sebbene l’analisi riguardi esclusivamente la documentazione fornita dalla controparte, non si esclude la possibilità di interpretazioni differenti rispetto a quanto prospettato. I fatti oggetto dell’indagine potrebbero essere letti in modi diversi a seconda delle circostanze.
Nel contesto dell’indagine, il Giudice ha autorizzato il sequestro del materiale informatico in uso presso Partolini, inclusi dispositivi di memorizzazione dati. Particolare attenzione è stata rivolta al computer portatile utilizzato dalla segretaria, che fungeva da intermediaria nei rapporti tra 23Bottles e Partolini.
Il materiale sequestrato è stato successivamente affidato ai consulenti tecnici delle parti per le opportune verifiche. Inoltre, i file contenuti nella memoria USB depositata da 23Bottles sono disponibili per l’analisi.
\vspace{14pt}\\
