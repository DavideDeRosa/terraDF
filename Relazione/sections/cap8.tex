Dall’analisi forense condotta sul materiale raccolto, è possibile trarre le seguenti conclusioni, che rafforzano la posizione difensiva di Partolini:

\begin{enumerate}
    \item \textbf{Validità delle Comunicazioni Ufficiali:}  
    le email scambiate tra 23Bottles e Partolini, registrate con le date e gli orari indicati dai server di posta elettronica, testimoniano una fase iniziale di trattativa commerciale regolare (ad es. [\ref{email1}], [\ref{email2}]). Tali comunicazioni ufficiali risultano affidabili, mentre le ulteriori email scambiate tramite indirizzi personali sembrano riferirsi a dinamiche interne non per forza riconducibili alle operazioni logistiche.

    \item \textbf{Evidenze di Accessi e Operazioni sul Sistema:}  
    l’analisi del collegamento di un hard disk esterno [\ref{wd}] in data 01/05/2023, unitamente ai log di sistema in \texttt{C:/Windows/System32/winevt/Logs/}, ha rivelato numerosi cambi di data e accessi al sistema, con variazioni temporali significative (ad esempio, un salto di 21 anni e retrocessi di mesi e giorni) effettuati dall’utente Laura.

    \item \textbf{Inaccessibilità del File Important Document.zip:}  
    il file presente nella directory \texttt{C:/Users/Laura/Desktop} mostra un’alta entropia (circa 7.999983 bits per byte) e risulta cifrato mediante algoritmi di full-text encryption (ad es. VeraCrypt o TrueCrypt). Nonostante i tentativi di decifrazione tramite \texttt{hashcat} e varie procedure di boot in ambiente virtuale, il contenuto del file è rimasto inaccessibile. Tale circostanza implica che il file non fornisce elementi probatori aggiuntivi a supporto delle tesi della parte attrice, impedendo una ricostruzione completa di eventuali discrepanze nella gestione del trasporto.
\end{enumerate}
Complessivamente, il materiale analizzato – includendo le comunicazioni ufficiali, i log di sistema e il file cifrato – non fornisce basi sufficienti per dimostrare un comportamento scorretto o un inadempimento contrattuale da parte di Partolini.\vspace{14pt}\\
Pertanto, alla luce di quanto emerso dall’indagine forense, si ritiene che il materiale evidenzi, pur la presenza di alcuni elementi anomali, la regolarità delle operazioni svolte da Partolini.\\In assenza di ulteriori prove chiare e verificabili, non è possibile attribuire a Partolini responsabilità per eventuali errori nella gestione del trasporto, rafforzando così la posizione difensiva dell’azienda.

