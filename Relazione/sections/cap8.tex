Alla luce degli elementi sopra esposti, è possibile trarre le seguenti conclusioni:

\begin{enumerate}
    \item \textbf{Incertezza sull'Integrità dei Dati:} Le evidenze di accessi remoti e le manipolazioni nelle date indicano che il materiale informatico su cui si basa il ricorso potrebbe non essere completamente attendibile. Sebbene le comunicazioni ufficiali riportino le date così come registrate dal server di posta elettronica, le anomalie riscontrate impediscono di ricostruire in maniera del tutto certa la cronologia degli eventi, minando in parte le affermazioni di 23Bottles.
    
    \item \textbf{Possibile Interferenza Esterna:} La presenza di anomalie legate agli accessi remoti suggerisce che operazioni di modifica potrebbero essere state effettuate da soggetti esterni, il cui intervento ha potenzialmente alterato il contenuto e la veridicità delle comunicazioni e dei documenti.
    
    \item \textbf{Ambiguità del File ``important documents'':} Il file inaccessibile rappresenta un punto critico, poiché la sua eventuale decodifica potrebbe far luce sulla corretta gestione del trasporto. La sua inaccessibilità, tuttavia, solleva sospetti circa un tentativo di occultamento o di manomissione dei dati, escludendo così la possibilità di utilizzarlo come prova contro Partolini S.r.l.
    
    \item \textbf{Esclusione della Responsabilità di Partolini S.r.l.:} Considerando i fatti sopra descritti, non vi è alcuna prova inconfutabile che Partolini S.r.l. abbia alterato in maniera dolosa o negligente il servizio contrattualmente pattuito. Le anomalie riscontrate nel materiale informatico, seppur parzialmente attribuibili ad interventi esterni e tentativi di manomissione, non dimostrano una mancanza di adempimento da parte di Partolini.
\end{enumerate}

\section*{Conclusione Generale}
Alla luce di tutte le evidenze raccolte, si ritiene che il materiale informatico depositato da 23Bottles presenti significative anomalie dovute a accessi remoti e manomissioni, in particolare l'alterazione delle date e l'inaccessibilità del file ``important documents''. Pur riconoscendo che le comunicazioni ufficiali riportano date e orari validi, come registrati dal server di posta elettronica, tali anomalie impediscono una ricostruzione affidabile e completa della vicenda. Ne consegue che le affermazioni di parte attrice non possano essere considerate pienamente attendibili e che non vi siano elementi probatori sufficienti a dimostrare un inadempimento contrattuale da parte di Partolini S.r.l.
