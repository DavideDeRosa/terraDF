\documentclass[a4paper,12pt]{report}
\usepackage[utf8]{inputenc}
\usepackage[T1]{fontenc}
\usepackage{geometry}
\usepackage[hidelinks]{hyperref}
\usepackage{graphicx}
\usepackage{titlesec}
\usepackage{fancyhdr}
\usepackage{lipsum}

\pagestyle{fancy}
\fancyhf{}
\fancyhead[L]{\textit{\chaptertitle}}
\fancyhead[R]{\thepage}
\renewcommand{\headrulewidth}{0.000009pt}

\fancypagestyle{plain}{
  \fancyhf{}
  \fancyhead[L]{\textit{\chaptertitle}}
  \fancyhead[R]{\thepage}
  \renewcommand{\headrulewidth}{0.4pt}
}

\newcommand{\chaptertitle}{}
\renewcommand{\chaptermark}[1]{
  \markboth{#1}{}
  \renewcommand{\chaptertitle}{#1}
}

\geometry{a4paper, margin=1in}

\titleformat{\chapter}[block]
  {\normalfont\LARGE\bfseries}{\thechapter.}{1em}{}
\titleformat{\section}[block]
  {\normalfont\Large\bfseries}{\thesection.}{1em}{}

\hypersetup{
    colorlinks=false,
    linkcolor=blue,
    filecolor=magenta,      
    urlcolor=cyan,
    pdftitle={Document},
    bookmarks=true,
    pdfpagemode=FullScreen,
}

\renewcommand{\contentsname}{Indice}

\begin{document}

\begin{titlepage}
    \centering
    \vspace*{0.1cm}

    \Huge
    \textbf{UNIVERSITÀ DI BOLOGNA}

    \vspace{1cm}
    \Large
    Dipartimento di Informatica - Scienza e Ingegneria \\
    Corso di Laurea Magistrale in Informatica \\\vspace{1cm}
    Progetto corso di \href{https://www.unibo.it/it/studiare/dottorati-master-specializzazioni-e-altra-formazione/insegnamenti/insegnamento/2024/479039}{Digital Forensics} \\
    A.A. 2024/25

    \vspace{5.5cm}
    \textbf{\LARGE 23Bottles S.p.A. vs Partolini S.r.l.}\\\vspace{0.3cm}
    \Large Relazione parte Convenuta

    \vfill

    \vfill

    \large
    Davide De Rosa \hfill Matricola: 0001186536\\
    Marco Coppola \hfill Matricola: 0001170490\\
    Valerio Pio De Nicola \hfill Matricola: 0001170425\\
\end{titlepage}

\tableofcontents
\newpage

\chapter{Informazioni amministrative}
L’azienda \textbf{23Bottles S.p.A.}, d’ora in poi \textit{23Bottles}, decide di presentare istanza di ricorso nei confronti dell’azienda di trasporti \textbf{Partolini S.r.l.}, d’ora in poi \textit{Partolini}, sostenendo che quest’ultima non abbia garantito il servizio di trasporto pattuito contrattualmente. Tale supposizione nasce da un recente scambio di e-mail tra 23Bottles e uno dei clienti, il quale afferma di aver ricevuto da Partolini un numero inferiore di unità dei prodotti di 23Bottles rispetto a quanto richiesto.\vspace{14pt}\\
L’azienda 23Bottles ha inoltre depositato al momento della presentazione della richiesta di ricorso, una \textit{memoria USB} contenente le e-mail scambiate con il cliente e le fatture fornite da Partolini. Questo materiale costituisce \textit{patrimonio aziendale riservato}, ed è l’oggetto del ricorso.\vspace{14pt}\\
Il Giudice ha autorizzato il sequestro del materiale informatico, qualsiasi supporto in grado di immagazzinare dati, in possesso dall’azienda Partolini. Si deve analizzare il materiale sequestrato, ossia il computer portatile detenuto dalla segretaria, che fungeva da tramite di 23Bottles e lo spedizioniere dell’azienda Partolini.\vspace{14pt}\\
Il materiale sequestrato è stato fisicamente consegnato ai consulenti tecnici delle parti. Il contenuto della memoria USB è reperibile presso il sito \textit{Virtuale} del corso.\vspace{14pt}\\
Il Giudice ha posto il seguente quesito, al quale i consulenti delle parti e del Giudice sono chiamati a rispondere:
\begin{quote}
    \textit{"Considerando che oggigiorno le aziende si avvalgono di una serie di
    servizi per la gestione delle proprie attività. Queste aziende possono essere
    vittime di frodi: in queste situazioni, investono risorse finanziarie in servizi
    che promettono beneici ma, al contrario, causano danni all’attività. 
    Si chiede agli specialisti forensi di veriicare la veridicità delle
    affermazioni di parte attrice sulla base del materiale depositato e quello
    sequestrato."}
\end{quote}

\pagebreak

\chapter{Sommario esecutivo}
ciao

\section{Obiettivo}
La presente relazione fornisce un’analisi dettagliata delle criticità emerse nella documentazione presentata dalla controparte durante il procedimento investigativo.\\
L’obiettivo è identificare eventuali discrepanze o incongruenze che potrebbero incidere sulla validità delle prove. Il documento è strutturato per esaminare in modo approfondito ogni elemento contestato, offrendo una valutazione basata su prove concrete.\vspace{14pt}\\
Sebbene l’analisi riguardi esclusivamente la documentazione fornita dalla controparte, non si esclude la possibilità di interpretazioni differenti rispetto a quanto prospettato. I fatti oggetto dell’indagine potrebbero essere letti in modi diversi a seconda delle circostanze.\\
Nel contesto dell’indagine, il Giudice ha autorizzato il sequestro del materiale informatico in uso presso Partolini, inclusi dispositivi di memorizzazione dati. Particolare attenzione è stata rivolta al computer portatile utilizzato dalla segretaria, che fungeva da intermediaria nei rapporti tra 23Bottles e Partolini.\\
Il materiale sequestrato è stato successivamente affidato ai consulenti tecnici delle parti per le opportune verifiche.\\
Inoltre, i file contenuti nella memoria USB depositata da 23Bottles sono disponibili per l’analisi.
\vspace{14pt}\\

\section{Metodologie}

L'indagine ha seguito tre fasi distinte:

\begin{enumerate}
    \item \textbf{Acquisizione USB}: inizialmente, è stata condotta un'analisi dei dati contenuti nella memoria USB fornita da 23Bottles, al fine di ottenere una panoramica completa degli eventi e delle comunicazioni. Tale attività ha previsto la sintesi delle email e l'analisi degli eventi associati. Durante questa fase, sono stati identificati gli indirizzi email di rilevanza ai fini della dinamica investigativa.\\Inoltre, si è resa necessaria l'analisi di email cifrate, che sono state successivamente decifrate per permettere un'indagine approfondita dei contenuti. Infine, è stato possibile ricostruire gli eventi in una linea temporale, garantendo una comprensione dettagliata della sequenza degli avvenimenti e delle interazioni tra le parti coinvolte.
    \item \textbf{Acquisizione HDD}: successivamente, è stata eseguita un'acquisizione forense dell'hard disk del computer utilizzato dalla segretaria dell'azienda Partolini, per garantire l'integrità dei dati raccolti e prevenire la contaminazione delle prove.
    \item \textbf{Analisi comparativa}: per concludere, è stata condotta un'analisi comparativa dei dati, confrontando le informazioni presenti nella memoria USB con quelle acquisite dal computer di Partolini. L'obiettivo di questa fase è stato quello di identificare corrispondenze o discrepanze, al fine di confermare o confutare le affermazioni avanzate da 23Bottles.
\end{enumerate}
Questo approccio ha permesso di condurre un'analisi forense rigorosa e completa, fornendo una panoramica dettagliata dei fatti e delle circostanze oggetto della controversia legale.


\pagebreak

\chapter{Acquisizioni forensi}
\section{PC Portatile Compaq}
In conformità all’autorizzazione del Giudice per il sequestro del materiale informatico appartenente all’azienda Partolini, questa sezione descrive il processo di analisi del materiale confiscato.\\
Tra gli elementi sequestrati vi è il computer portatile utilizzato dalla segretaria, figura chiave nell’intermediazione tra l’azienda Partolini e 23Bottles.\vspace{14pt}\\
Per acquisire l’immagine forense del computer, abbiamo inizialmente smontato il dispositivo ed estratto il disco. Abbiamo successivamente collegato il write blocker -- modello \textit{Tableau TK8U} --  al computer e successivamente il disco al write blocker, seguendo la sequenza prestabilita. Una volta alimentati i dispositivi, abbiamo avviato \textbf{FTK Imager}\footnote{FTK Imager, strumento gratuito fornito da \textit{AccessData}, offre diverse funzionalità, tra cui la possibilità di generare un valore hash per il dispositivo sorgente, creare un’immagine forense e successivamente calcolare un altro valore hash per verificare l’integrità del processo, assicurando l’assenza di modifiche al dispositivo originale.}.\vspace{14pt}\\
Abbiamo eseguito una verifica preliminare dell’integrità del dispositivo tramite FTK Imager. Il valore hash ottenuto ha confermato che non erano state apportate modifiche al dispositivo sorgente, permettendoci così di procedere con la creazione dell’immagine forense.\vspace{14pt}\\
Per il formato dell’immagine, abbiamo scelto il \textbf{formato E01}, considerato il più adatto alle esigenze investigative del caso.

\section{Chiavetta USB}
La chiavetta USB fornita da 23Bottles è stata acquisita utilizzando FTK Imager come volume logico, permettendoci di ottenere anche i diversi Hash dei documenti presenti al suo interno.\vspace{14pt}\\
Viene ignorato il contenuto della cartella \_\_MACOSX, directory di sistema aggiunta automaticamente da MacOS quando si comprime o archivia un insieme di file.

\pagebreak

\section{Narrativa}
La narrativa rappresenta la "storia" dell'indagine, spiegando cosa è successo, cosa è stato fatto e quali conclusioni sono state tratte.

\subsection{Cosa dovrebbe contenere la narrativa?}
\begin{itemize}
    \item \textbf{Executive Summary (Sintesi Esecutiva)}: panoramica dei punti chiave e delle conclusioni.
    \item \textbf{Dettagli amministrativi}: elenco delle persone coinvolte (sospetti, vittime, testimoni, investigatori, ecc.).
    \item \textbf{Fatti e circostanze}: sequenza degli eventi che ha portato all'indagine, incluse sfide affrontate e soluzioni adottate.
    \item \textbf{Elenco delle prove}: dispositivi esaminati con dettagli tecnici (modello, numero seriale, stato, ecc.).
\end{itemize}

\subsection{Immagini e screenshot}
Screenshot e immagini devono essere inclusi solo se spiegati chiaramente, evidenziando la loro rilevanza per il caso.

\section{Esibizioni Pertinenti (Evidenze)}
Questa sezione descrive in dettaglio le prove digitali raccolte durante l'indagine.

\subsection{Cosa dovrebbe contenere questa sezione?}
\begin{itemize}
    \item \textbf{Dettaglio delle prove trovate}: file, immagini o dati specifici e la loro posizione.
    \item \textbf{Ordine cronologico}: presentazione degli artefatti in ordine temporale per mostrare l'evoluzione degli eventi.
    \item \textbf{Descrizione degli artefatti}: spiegazione obiettiva di ogni prova, evitando opinioni personali.
\end{itemize}

\section{Documentazione di Supporto}
Sezione dedicata alle informazioni tecniche e aggiuntive che garantiscono trasparenza e affidabilità.

\subsection{Cosa include questa sezione?}
\begin{itemize}
    \item \textbf{Glossario}: definizione dei termini tecnici utilizzati nel report.
    \item \textbf{Processo di acquisizione delle immagini forensi}: descrizione del metodo utilizzato per creare copie forensi dei dispositivi.
\end{itemize}

\section{Conclusione}
La conclusione riassume le principali scoperte e riflessioni finali.

\subsection{Cosa includere nella conclusione?}
\begin{itemize}
    \item \textbf{Sintesi delle prove e dei risultati}: riassunto obiettivo delle evidenze raccolte.
    \item \textbf{Riflessioni obiettive}: evidenziare eventuali incertezze o limiti delle conclusioni senza fare affermazioni non supportate da prove.
\end{itemize}

\section{Formattazione e Firma Digitale}
\begin{itemize}
    \item \textbf{Firma digitale}: garantisce l'integrità del documento.
    \item \textbf{Revisione}: il report deve essere sottoposto a rilettura per individuare errori o potenziali ambiguità.
\end{itemize}

\end{document}